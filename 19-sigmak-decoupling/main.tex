
\documentclass{article}
\usepackage{nips_2018,graphicx}

 \usepackage[english]{babel} 
  \usepackage{times}		
\usepackage[T1]{fontenc}
\usepackage[utf8]{inputenc} 
\DeclareUnicodeCharacter{00A0}{~}
\usepackage{amssymb,amsmath,amscd,amsfonts,amsthm,bbm,mathrsfs,yhmath}
\usepackage[shortlabels]{enumitem}
\usepackage{hyperref}

\usepackage[textwidth=2cm, textsize=footnotesize]{todonotes}  
\setlength{\marginparwidth}{1.5cm}               %  this goes with todonotes
\newcommand{\aknote}[1]{\todo[color=cyan!20]{#1}}
\newcommand{\asnote}[1]{\todo[color=green!]{#1}}

\newcommand{\cF}{{\mathcal F}} 
\newcommand{\cS}{{\mathcal S}} 
\newcommand{\cM}{{\mathcal M}} 
\newcommand{\cP}{{\mathcal P}} 

\newcommand{\cC}{{\mathcal C}} 
\newcommand{\E}{{{\mathbb E}}} 
\newcommand{\bR}{{\mathbb R}} 
\newcommand{\bP}{{\mathbb P}} 
\newcommand{\bE}{{\mathbb E}} 
\newcommand{\sX}{{\mathsf X}} 


\newcommand{\KL}{\mathop{\mathrm{KL}}\nolimits}
\newcommand{\tr}{\mathop{\mathrm{tr}}\nolimits}
\newcommand{\ps}[1]{\langle #1 \rangle}
\newcommand{\Supp}{\mathop{\mathrm{Supp}}\nolimits}


\theoremstyle{definition}
\newtheorem{theorem}{Theorem}
\newtheorem{lemma}[theorem]{Lemma}
\newtheorem{corollary}[theorem]{Corollary}
\newtheorem{proposition}[theorem]{Proposition}
\newtheorem{definition}{Definition}
\newtheorem{remark}{Remark}
\newtheorem{condition}{Condition}
\newtheorem{assumption}{Assumption}
\newtheorem{example}{Example}

\title{From $\sigma_k$ to Decoupling}


\begin{document}

\maketitle

\begin{abstract} 

\end{abstract}

Generally, the goal of this work is to go from the analysis of SGD proposed in the $\sigma_k$ paper~\cite{gorbunov2019unified} to the decoupling paper~\cite{mishchenko2019stochastic}.

In~\cite{gorbunov2019unified}, SGD is analyzed under the following assumptions 
\begin{assumption}
Unbiasedness
\begin{equation}
    \bE[g^k|x^k] = \nabla f(x^k)
\end{equation}
\end{assumption}

We are especially trying to understand the following points.
\begin{itemize}
    \item \cite{gorbunov2019unified} provide a tight analysis of SGD, can we generalize it to the setting of~\cite{mishchenko2019stochastic}
    \item If we particularize~\cite{mishchenko2019stochastic} to the setting of SGD, what kind of analysis do we get? Especially, it allows biased estimate of the gradient.
    \item In~\cite{mishchenko2019stochastic}, what is the assumption made on $v_t$ ? what is $Y_t$?
    \item Minibatch analysis of~\cite{mishchenko2019stochastic} 
\end{itemize}



\subsubsection*{References}
\renewcommand\refname{\vskip -1cm}
\bibliographystyle{apalike}
\bibliography{biblio,math}

\end{document}